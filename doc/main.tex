\documentclass{report}

\usepackage{amsmath,amssymb}
\usepackage{graphicx}
\usepackage{booktabs}
%\usepackage{chemfig}

\newcommand{\pd}[2]{ \frac{\partial #1}{\partial #2} }
\newcommand{\order}[1]{ \mathcal{O}\left( \ensuremath{#1} \right) }
\newcommand{\m}[1]{ \mathrm{#1} }

\begin{document}

\chapter{Biogeochemical model}
\section{Model scope}
We treat the hypolimnion as a mostly closed system of cycling nutrients. Compartments are linked by transport processes, and reactions within compartments interconvert simulated chemical species. Interaction with the environment outside the hypolimnion is represented by just two processes: input of oxidizable carbon in the upper compartments and input of methane from the sediment. The sediment and metalimnion are otherwise ignored.

\section{Model}
\subsection{Chemical species and reactions}
We use a simplified set of chemical species to avoid excessive parameterization and
because we need only predict values for which we have experimental comparison. Table
\ref{tab:chemical_species} lists all chemical species simulated.

In nature, carbon is added to the hypolimnion through particulate biomass, humics from rainwater runoff, photosynthesis, etc. There are many carbon species with differing oxidation states. In this model, we simulate only one species of carbon (C) that has only one oxidation state.

In the model, many chemical compounds are considered abundant or unimportant to the simulated biogeochemical cycles and are therefore excluded. Carbon dioxide is considered ubiquitous and abundant. Its production and consumption are not tracked and reactions that consume
CO$_2$ are assumed to proceed without interruption. Nitrogen gas, produced by denitrification
and iron oxidation on nitrate, is ignored.

The pH in Mystic Lake's hypolimnion varies between about 6 and 7. We neglected effects of pH on chemical species formation.

\begin{table}
\centering
\begin{tabular}{ l l l }
\toprule
symbol & name  & representative compounds \\
\midrule
O   & dissolved oxygen    & O$_2$ \\
C   & oxidizable carbon & cyanobacteria biomass, glucose, acetate \\
N$^+$ & oxidized nitrogen &  nitrate, nitrite \\
N$^-$ & reduced nitrogen  & ammonia \\
Fe$^+$ &  oxidized iron  & Fe(III) compounds  \\
Fe$^-$ &  reduced iron  & Fe(II)  \\
S$^+$ &  oxidized sulfur  & sulfate compounds \\
S$^-$ &  reduced sulfur  & sulfide compounds  \\
M   &   methane &   CH$_4$ \\
\bottomrule
\end{tabular}
\caption{Chemical species included in the model.}
\label{tab:chemical_species}
\end{table}

\subsection{Inferred microbial biomass}
Typically an implicit biomass model is used when the abundance of microbes catalyzing
simulated reactions is irrelevant to the study. Explicit biomass models, in contrast, require parameterizing the growth kinetics for each modeled microbial species and so typically only include a few biological species. 

Rather than predict the observable abundance of a few chemical species, here we focus on the
distribution of bacteria performing similar metabolisms. If each organism performed only
one metabolism as defined by the reactions in the model, all organisms performing that
metabolism would group into an ecological guild. We assume that the
rate of a reaction at a given depth is proportional to the number of individuals in the guild with the
corresponding metabolism in that spatial compartment:
\begin{equation}
  (\text{reaction rate}) = (\text{rate per individual}) \times (\text{number of guild members}).
\end{equation}
Even if the constant of proportionality (rate per individual) is not known, the distribution of
that guild's biomass across
depth can be inferred from the relative reaction rates at each depth.

As noted in Hunter et al.\cite{hunterkinetic1998}, an implicit biomass model is equivalent to assuming that population
biomass is only limited by the availability of the energy source, that is, the guild
quickly equilibrates to the size of the available niche.

\section{Mechanics: Transport and reactions}
The rate of change in the concentration of a chemical species $X$ at a depth $i$ is
\begin{equation}
  \frac{\partial X_i}{\partial t} = \left( \text{transport terms} \right) + \left(
  \text{reaction terms} \right) + \left(\text{source terms}\right),
\end{equation}
where low $i$ refers to low depth in meters, i.e., vertically higher in the water column.
The simulation proceeds in $N$ compartments, which we spaced at one meter to be comparable
to the collected chemical and biological data. The initial concentrations are set and the
simulation proceeds for a time $T$, during which the chemical species concentrations and
reaction rates are recorded. This time roughly corresponds to the period between the
movement of the thermocline up the water column in spring and the breakdown of
stratification in fall.

\subsection{Transport: Diffusion and precipitation}
Most chemical species are treated as dissolved in the water column. In the time and length scales relevant to the hypolimnion ecosystem, molecular diffusion is slow compared to bulk transport processes like vertical eddy diffusion. To model these bulk transport processes, most chemical species are transported by simple diffusion with rate $D \left(X_{i-1} - X_i\right) + D \left(X_{i+1} - X_i \right)$, where the diffusion constant $D$ is the same for all chemical species, since it represents a bulk transport process. To account for the boundaries at the metalimnion and sediment, the first term is excluded in the uppermost simulation compartment; in the lowermost compartment, the second is excluded.

To simulate the precipitation of particulate carbon and oxidized iron species, Fe$^+$ and C precipitate in the model. A parameter $p$, where $0 < p < 1$,
determines the balance between vertical eddy diffusion and precipitation for these chemical species so that
the transport rate is
\begin{equation}
  (1 + p) D \left(X_{i-1} - X_i\right) - (1-p) D \left(X_{i+1} - X_i\right).
\end{equation}
Since $p > 0$, these species tend to move down the water column and accumulate above the sediment. As with other species, the first
term in excluded in the top compartment; the second term in the bottom compartment.

\subsection{Reactions}
\subsubsection{Biotically-catalyzed reactions: Primary oxidations}
The oxidation of carbon uses a chain of progressively less energetically-favorable terminal
electron acceptors. Here, we follow the formulation laid out by Hunter et
al.\cite{hunterkinetic1998}, equations 3 and 4.

The total rate of carbon degradation in a compartment follows first order kinetics:
\begin{equation}
  R^\m{C} \equiv k^\m{C} \m{C}; \quad \left( \frac{\partial \m{C}}{\partial t}
  \right)_\text{reaction} = -R^\m{C}.
\end{equation}
The fraction of carbon taken up by oxidation on each of the terminal electron acceptors is
determined by the abundance and relative metabolic merit of the electron acceptors. The $j$-th electron acceptor is
consumed at a rate
\begin{equation}
  R_j = \frac{f_j}{e_j} R^\m{C},
\end{equation}
where $e_j$ is the number of electrons neutralized per electron acceptor molecule and $f_j$
is determined by successive applications of the formula
\begin{equation}
  f_j = \left( 1 - \sum_{k=1}^{j-1} f_k \right) \mathrm{max} \left\{ 1,
  \frac{[\mathrm{EA}_j]}{[\mathrm{EA}_{\mathrm{lim},j}]} \right \}
\end{equation}
for $j \in \{ 1 \ldots 4 \}$. If the $j$-th electron acceptor's concentration $[\mathrm{EA}_j]$ is greater than some constant
limiting concentration $[\mathrm{EA}_{\mathrm{lim},j}]$, then that electron acceptor gets
all the remainder of the carbon; otherwise, it gets a fraction of what is left determined by
the ratio of the two concentrations.

The electron acceptors and their $e_j$ are listed in Table \ref{tab:electron_acceptors}. Methanogenesis
corresponds to $j=5$, and gets all remaining carbon so that $f_5 = 1 - \sum_{k=1}^4 f_k$.
All the carbon allocated by $R^\m{C}$ gets used up (i.e., $\sum_{j=1}^5 = 1$), but each electron
acceptors accepts electrons according to a different stoichiometry (i.e., $\sum_{j=1}^5 R_j \ne
R^\m{C}$).

\begin{table}
\centering
\begin{tabular}{ l l l }
\toprule
$j$ &   EA   &   $e_j$   \\
\midrule
1   &   O   & 4  \\
2   &   N$^+$   & 5  \\
3   &   Fe$^+$  & 1  \\
4   &   S$^+$   & 8  \\
5   &   $\varnothing$   & 8  \\
\bottomrule
\end{tabular}
\caption{Electron acceptors in the primary oxidation reactions. $j=5$ corresponds to
methanogenesis.}
\label{tab:electron_acceptors}
\end{table}

\begin{table}
\begin{tabular}{ l l l l }
\toprule
Primary oxidations  & rate \\
\midrule
$\m{C} \rightarrow a \m{N}^- + e \m{e}^-$   & $R^\m{C}$ & primary oxidation half-reaction \\
$\m{O} \to \varnothing$    &   $R_1$   & aerobic heterotrophy \\
$\m{N}^+ \to \varnothing$    &   $R_2$   & denitrification \\
$\m{Fe}^+ \to \m{Fe}^-$ & $R_3$ & iron reduction  \\
$\m{S}^+ \to \m{S}^-$ & $R_4$ & sulfate reduction  \\
$\varnothing \to \m{M}$ & $R_5$ & methanogenesis \\
\\
\midrule
Biotic secondary oxidations & rate constant \\
\midrule
$2 \m{O} + \m{N}^- \rightarrow \m{N}^+$  & $k_1$ & ammonia oxidation   \\
$2 \m{O} + \m{S}^- \rightarrow \m{S}^+$  & $k_2$ & sulfide oxidation   \\
$\m{N}^+ + 5 \m{Fe}^- \rightarrow 5 \m{Fe}^+$    &   $k_3$  & iron oxidation on nitrate  \\
$\m{CH}_4 + 2 \m{O} \rightarrow \varnothing$    &   $k_4$  & methanotrophy on oxygen  \\
$\m{CH}_4 + \m{S}^+ \rightarrow \m{S}^-$    &   $k_5$  & methanotrophy on sulfate  \\
\\
\midrule
Abiotic secondary oxidations    & rate constant \\
\midrule
$\tfrac{1}{4} \m{O} + \m{Fe}^- \rightarrow \m{Fe}^+$ & $k_6$ & iron oxidation  \\
\bottomrule
\end{tabular}
\label{tab:reactions}
\caption{Reactions simulated in the model.}
\end{table}

\subsection{Secondary oxidations}
We model secondary oxidations, the oxidation of compounds other than carbon compounds, using
second-order mass action kinetics as per Hunter et al.\cite{hunterkinetic1998} (Table 4). For the
transformation of substrates $S_1, S_2$ into a product $P$ according to $a_1 S_1 + a_2 S_2
\to b P$, the reaction rate is $r \equiv k [S_1] [S_2]$ and the reaction terms are
\begin{align}
  \left( \pd{[P]}{t} \right)_\text{reaction} &= b r \\
  \left( \pd{[S_i]}{t} \right)_\text{reaction} &= -a_i r \quad \left( i = 1, 2 \right) \\
\end{align}
with rate constant $k$. As per Hunter et al., we do not adjust the rate according to the reaction's stoichiometry.

Primary and second oxidations are listed in Table \ref{tab:reactions}.

\subsection{Source terms}
Interactions between the hypolimnion and the outside world are modeled by simple source terms. Oxygen and carbon are added at the thermocline. Methane can be produced by primary oxidation in the water column, but methanogenesis also proceeds in the sediment, where it is transported upward and consumed by methanotrophy. We model this process by a point source of methane in the sediment.
All methane in our model is consumed before reaching the thermocline, so we omit the mechanics for emission of methane into the metalimnion.

\subsection{Parameterization}
A list of parameters and their values is included somewhere around here. Most parameters related to the reaction rates are borrowed from Hunter et al.\cite{hunterkinetic1998}. The parameters related to transport, sink/source terms, and initial concentrations were drawn from sources where possible. Some values were adjusted by hand to match the observed data. 

\begin{table}
\centering
\begin{tabular}{ l r l l }
\toprule
parameter & value & unit & source \\
\midrule
\multicolumn{4}{l}{General parameters} \\
$T$     & 1 & yr & Asserted to set time scale \\
$N$     & 17 & & Asserted to set compartment height to 1 m   \\
\\
\multicolumn{4}{l}{Primary oxidation parameters} \\
$a$ (N:C ratio)	&	0.1	& & See below \\
$\left[\mathrm{O}_\mathrm{lim}\right]$ & 20.0 & $\mu$M & HWcV 20.0 \\
$\left[\mathrm{N}^+_\mathrm{lim}\right]$ & 5.0 & $\mu$M & HWcV 5.0 \\
$\left[\mathrm{Fe}^+_\mathrm{lim}\right]$ & 0.1 & $\mu$M & HWcV 60 umol dm-3 \\
$\left[\mathrm{S}^+_\mathrm{lim}\right]$ & 30.0 & $\mu$M & HWcV 30.0 \\
\\
\multicolumn{4}{l}{Secondary oxidation parameters} \\
$k_1$ (ammonia oxidation)	& $5.0$	& $\mu$M$^{-1}$ yr$^{-1}$ & HWvC $k_{4}^\mathrm{sr} = 5.0$ \\
$k_2$ (sulfide oxidation)	& $0.16$	& $\mu$M$^{-1}$ yr$^{-1}$ & Manually adjusted \\
$k_3$ (iron oxidation, nitrate)	& $0.01$	& $\mu$M$^{-1}$ yr$^{-1}$ & Manually adjusted \\
$k_4$ (methanotrophy, oxygen)	& $10^{3}$	& $\mu$M$^{-1}$ yr$^{-1}$ & $10^{4}$ ($k_{9}^\mathrm{sr}$) \cite{hunterkinetic1998} \\
$k_5$ (methanotrophy, sulfate)	& $10^{-3}$	& $\mu$M$^{-1}$ yr$^{-1}$ & HWvC $k_{10}^\mathrm{sr} = 10^{-2}$ \\
$k_6$ (iron oxidation) & $10^4$ & $\mu$M yr$^{-1}$ & HWvC $k_{2}^\mathrm{sr} = 10^{-1}$ \\
\\
\multicolumn{4}{l}{Transport parameters} \\
$D$	&	50	&	m$^2$ yr$^{-1}$	&	See below \\
$p_\mathrm{Fe}$	&	0.3	&	& Manually adjusted \\
$p_\mathrm{C}$	&	0.1	&	& Manually adjusted \\
\\
\multicolumn{4}{l}{Source rates} \\
$s_\mathrm{C}$	&	$4.5 \times 10^4$ & $\mu$M yr$^{-1}$	&	Manually adjusted \\
$s_\mathrm{O}$	&	$3.5 \times 10^3$ & $\mu$M yr$^{-1}$	&	Manually adjusted \\
$s_\mathrm{M}$	&	1500 & $\mu$M y$^{-1}$	&	See below \\
\\
\multicolumn{4}{l}{Initial concentrations} \\
C	&	200 & $\mu$M \\
O	&	50 & $\mu$M \\
N total	&	100 & $\mu$M \\
N ratio	&	10 & $\mu$M \\
Fe tot	&	60 & $\mu$M \\
Fe rat	&	10 & $\mu$M \\
S tot	&	100 & $\mu$M \\
S rat	&	10 & $\mu$M \\
\bottomrule
\end{tabular}
\label{tab:parameters}
\caption{Parameter values and sources.}
\end{table}

\subsubsection{N:C ratio $a$}
USGS data for the Aberjona river gauge (citation?, which drains into Mystic Lake) for 1999-2000 has average total organic carbon 5.59 ml L$^{-1}$. Wetzel (\cite{Wetzel_2001}, Table 12-4, pg. 224) lists C:N for this carbon amount 15.1, thus N:C $= 1/15.1 = 0.066$.

\subsection{Eddy diffusion constant $D$}
Benoit and Hemond\cite{benoitvertical1996} collate reports of vertical eddy diffusion constants 0.002--0.05 cm$^2$ s$^{-1}$ for lakes with depths comparable to Mystic Lake's, corresponding to a range 6--158 m$^2$ yr$^{-1}$.

\subsection{Methane source $s_\mathrm{M}$}
Varadharajan thesis pg 202 reports 1.3--4.0 mmol m$^{-2}$ d$^{-1}$, which corresponds to 475--1460 $\mu$M yr$^{-1}$ if the methane is delivered to a well-mixed one-meter high lowest compartment as assumed in the model.

\section{Implementation}
The model was implemented in Matlab, and the ODE solutions were computed using the command {\tt ode15s} with all chemical species restricted to nonnegative values (command {\tt odeset}).

\bibliographystyle{plain}
\bibliography{refs}


\end{document}
